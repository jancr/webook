\documentclass[]{report}   % list options between brackets
\usepackage{xcolor}
\usepackage{listings}
\lstset
{ %Formatting for code in appendix
    %basicstyle=\footnotesize,
    % line number stuff
    numbers=left,
    stepnumber=1,
    xleftmargin=4.0ex,
    showstringspaces=false,
    tabsize=4,
    breaklines=true,
    breakatwhitespace=false,
}

\usepackage{textcomp}
\lstset{basicstyle=\ttfamily,
  showstringspaces=false,
  commentstyle=\color{red},
  keywordstyle=\color{blue}
}
\usepackage[utf8]{inputenc}
\usepackage[english]{babel}
 
\usepackage{hyperref}
\hypersetup{
    colorlinks=true,
    linkcolor=blue,
    filecolor=magenta,      
    urlcolor=cyan,
}
 
\usepackage{courier}
% type user-defined commands here


\definecolor{titlebgdark}{RGB}{0,163,243}
\definecolor{titlebglight}{RGB}{191,233,251}

%\usepackage{mystyle} % Nannas hacks
\begin{document}
					
\lstset{language=Python,upquote=true}

\title{WeBook: scrape Webpage to Ebook}
\author{Jan Christian Refsgaard}
\date{December 20, 2017}    % type date between braces
\maketitle

%-------------------------------------------------------------------------------
% ABBREVIATIONS
%-------------------------------------------------------------------------------
\section*{Abbreviations}
\begin{tabular}{ l l }
	% TODO: remember to sort them !!!!
	% TODO: remember to sort them !!!!
	\textbf{URI}   & \textbf{U}niform \textbf{R}esource \textbf{I}dentifier \\
	\textbf{URL}   & \textbf{U}niform \textbf{R}esource \textbf{L}ocator \\
	\textbf{EPUB}  & \textbf{E}lectronic \textbf{PUB}lication  \\
	\textbf{HTML}  & \textbf{H}yper\textbf{T}ext \textbf{M}arkup \textbf{l}anguage \\
	\textbf{XHTML} & \textbf{E}xtensible \textbf{H}yper\textbf{T}ext 
					 \textbf{M}arkup \textbf{l}anguage \\
	\textbf{HTML5} & \textbf{H}yper\textbf{T}ext \textbf{M}arkup \textbf{l}anguage 
					 version 5 \\
	\textbf{OEBPS} & \textbf{O}pen \textbf{EB}ook \textbf{P}ublication \textbf{S}tructure \\
\end{tabular}

%-------------------------------------------------------------------------------
% INTRODUCTION
%-------------------------------------------------------------------------------
\chapter{Introduction}           
Many aspiring authors publish their work on as a wordpress wordpress blog, two
of my favorite books are only available this way:
\begin{itemize}
    \item Worm: \url{parahumans.wordpress.com/}.
    \item From The New World: \url{shinsekai.cadet-nine.org/} 
\end{itemize}
There are other webpages such as \url{fanfiction.net} where users have uploaded millions of books.

Because the above books are only available as websites they cannot be consumed
from an e-book reader. Nor can they be enjoyed when internet is unavailable such
as on an air plane.

\section{Problem Statement}
How can a website be converted to an e-book?

This question can be further broken down to:
\begin{itemize}
    \item how do you scrape a website
    \begin{itemize}
        \item do you have to write a scraper for each website, or can the software be made more generic?
    \end{itemize}
    \item which e-book format resemble a website most?
    \item should the software support multiple e-book formats?
    \item How do we make the tool user friendly.
\end{itemize}

\subsection{Scope}
Many e-book format exists, but EPUP is based on html, and therefore the easiest
to support when the source book is also HTML. I will not support other formats
as Calibre \url{https://calibre-ebook.com/}, support conversion between most
e-book formats.

As there are a plethora of different webpage books, that all will be needed to
be converted to EPUP, I will only write a scraper for \url{fanfiction.net}, but
strive to make the code modular so it is easy for tech savvy users to write a
parser, to further facilitate the expansion of the tool, the code is available
on github under \url{https://github.com/jancr/webook} under the MIT licence. 

To make the tool available I will make a website, as well as a command-line
interface.

My Hope is for my tool to be used by other readers and programmers. Thus to
make installation as easy as possible I will strive to mostly use functions
from Pythons standard library.

%\subsection{Methods}

%-------------------------------------------------------------------------------
% THEORY
%-------------------------------------------------------------------------------
\chapter{Theory}
\section{Electronic Publication (EPUB)}
\subsection{EPUB History}
EPUB is the successor of the Open eBook Publication Structure (OEBPS) version
1.2. For this reason the first version of ePub, version 2.0, was released in
2007. In 2010 it reached maturity and the final specification EPUB 2.0.1 was
releasesd \ref{epub201}. This promptet the development of version 3 where there
is still active development\ref{epub301}. The Difference between the 2.0 and
3.0 mostly new features such as support for mathML and support for Fixed Layout Documents such as comments\ref{epub2to3}. Also version 2.0 uses XHTML where version 3.0 supports both XHTML and HTML5.EPUB 3.0.1 is mostly backwards compatible with 2.0.1\ref{epub2to3}. 

\subsection{Implementation}
As many e-eook readers have different screen sizes, the standard has opted to
use HTML + css, to facilitate reflowable documents.

An EPUB is basically a website, contained in a ZIP archive. To see this lets download a public domain book (note I am using the back tick hack to split the url over multiple lines):

\begin{lstlisting}[language=bash]
wget "https://s3-us-west-2.amazonaws.com/"`
     `"pressbooks-samplefiles/MetamorphosisJacksonTheme/"`
     `"Metamorphosis-jackson.epub"
unzip Metamorphosis-jackson.epub
\end{lstlisting}

Which yields the following output:

\begin{lstlisting}[language=bash]
Archive:  Metamorphosis-jackson.epub
 extracting: mimetype
  inflating: toc.ncx
  inflating: OEBPS/chapter-001-chapter-i.html
            ...
  inflating: OEBPS/jackson.css
  inflating: book.opf
  inflating: META-INF/container.xml
  inflating: META-INF/com.apple.ibooks.display-options.xml
\end{lstlisting}
% removed from ...
  % inflating: OEBPS/chapter-002-chapter-ii.html
  % inflating: OEBPS/title-page.html
  % inflating: OEBPS/front-cover.html
  % inflating: OEBPS/chapter-003-chapter-iii.html
  % inflating: OEBPS/copyright.html
  % inflating: OEBPS/table-of-contents.html
  % inflating: OEBPS/assets/pressbooks-promo.png
  % inflating: OEBPS/assets/MedulaOne-Regular.ttf
  % inflating: OEBPS/assets/themetamorphosis_1200x1600.jpg
  % inflating: OEBPS/pressbooks-promo.html

We see above that the website aspects of the e-book is located in the folder
\texttt{OEBPS}, while this is not mandatory, it is common practice, the folder
is named \texttt{OEBPS} because that is the predecessor to EPUB.

Apart from the website aspects of \texttt{Metamorphosis-jackson.epub} there
three other files mandated by the EPUB standard\ref{epub301}:
\begin{itemize}
    \item \texttt{META-INF/container.xml} This XML file points to the file OPF file(s)
    \item \texttt{book.opf}: Open Package Format (URI:
          \url{http://www.idpf.org/2007/opf}), a file containing meta data, such as
          author name, and a manifest indexing the books content, in this case all
          the files in OEBPS)
    \item \texttt{toc.ncx} The table of context, an XML file (URI:
          \url{http://www.daisy.org/z3986/2005/ncx/}.
\end{itemize}

\section{Python 3}
\subsection{Quick Python Intro for Programmers}
Python is one of the easiest languages to learn. The standard hello world is only one line:
\begin{lstlisting}[language=python]
print("hello world")
\end{lstlisting}

Python uses whitespace to signify end of code block, thus all code that have the same indentation are in the same block. This avoids the dangling else problem\ref{dangling_else}:

\begin{lstlisting}[language=python]
y = x = 2
if y == 3:
    if x == 3:
        print('x is 3')
else:  # the else belongs to the if with the same indentation
    print('y is not 3')
\end{lstlisting}

% \begin{lstlisting}[language=c]
% int y = 2
% int y = 2
% if (y == 3)
    % if (x == 3)
        % printf('yay')
% else 
    % printf('does this else belong to the y or x if statement??')
% \end{lstlisting}
% 
Python 3 is under rapid development, and in version 3.6 f strings were
introduced, allowing you to seamlessly intermingle code and strings, by
prepending stings with an \textbf{\texttt{f}}, you can run arbitrary python
code between \{ and \}:

\begin{lstlisting}[language=python]
x = 'hat,cat'
print(my f"{x.split(',')[0]} is beautiful")
# prints "my hat is beautifull"
\end{lstlisting}

For a more in depth tutorial the tutorial at \url{
https://docs.python.org/3/tutorial/index.html} is excellent.

%\subsection{OOP}
\subsection{Web Scraping}
Web scraping referes to the act of extracting data from a website, it can be
done manually, but most often it is done programmically. As crawling a server
risks overloading it, many pages have a file named
\texttt{robots.txt}\ref{robots} that specifies, which parts of the site robots
are allowed to scrape. Of course malicious bots does not have to respect the
\texttt{robots.txt}.

There are many different tools and libraries that can assist one. The most
simple way to scrape a website is simply to make GET requests to web pages,
find links, and then make GET requests to those. While this works wonders for
static pages, it falls short on dynamic webpages. For very advanced web
scraping libraries like Java's HtmlUnit, "a GUI less browser for Java
Programs"\ref{java_htmlunit}" be used to scrape dynamic pages.

\subsection{Flask}
Flask is a microframework, it is thus very lightweight and ideal for small
webpages. It builds upon two other python packages. 1. Werkzeug, a WSGI (Web
Server Gateway Interface). 2) Jinja 2, a templating language that allows you to
mix python and html, much like php. 3) 

Much like python, hello world in Flask is also surprisingly few lines:

\begin{lstlisting}[language=python]
import flask
app = flask.Flask(__name__)

@app.route("/")
def hello():
    return "Hello World!"
\end{lstlisting}

\textbf{Line 1-2:} Imports Flask, and instanciates the WSGI server \\
\textbf{Line 4:} this decorator assigns the '/' request (both GET and POST) to
the function \texttt{hello} \\
\textbf{Line 5-6:} a Python functio that returns "HTTP/1.1 200 OK" with "Hello
World" as data \\

If instead of returning plain text we want to return html, then we need two
things, 1. we need the function \texttt{flask.render\_template}, that takes a Jinja 2 template as argument and processes it into html, 2. A Jinja 2 template

\subsubsection{Jinja 2}
The base idea of Jinja is that you create one base template, and extend it,
data is passed to the template via \texttt{render\_template}. 


\begin{lstlisting}[language=python]


  <ul>
  
    <li><a href="{{ user.url }}">{{ user.username }}</a></li>
  
  </ul>

\end{lstlisting}

\textbf{Line 1:} This an extension of the template \texttt{layout.html} \\
\textbf{Line 2 and 8:} The following code goes into the 'body' block of
\texttt{layout.html} \\
\textbf{Line 4-6:} python code is encapsulated in \texttt{\{\% \%\}} and python
variables in \texttt{\{\{ \}\}}

Assuming the above template is called \texttt{users.html} and you want to pass
it the \texttt{user} variable, then render\_template would be called with two
arguments, 1. The template (\texttt{users.html}, and 2. The user
variable(\texttt{user})

\begin{lstlisting}[language=python]
@app.route("/users.html")
def users():
    return flask.render_templates('users.html', user=user)
\end{lstlisting}




%-------------------------------------------------------------------------------
% WEBOOK 
%-------------------------------------------------------------------------------
\chapter{WeBook}

\subsection{Design and Technology Choices}
\subsubsection{E-book format}
Because of EPUB 3.0.1 is mostly backwards compatible with EPUB
2.0.1\ref{epub2to3} and because the EPUB 2.0.1 specification is simpler, I have
opted for creating EPUB that conform to both standards. This is achieved by
using XHTML instead of HTML5.

\subsubsection{Python and Libraries}
To keep the project simple, I have opted to use the same language (Python) for:
scraping the webpages, creating the e-book, exposing the command line
interface and creating web server.
\textbf{E-Book generation:} As the EPUB is a mixture of xhtml and xml, a xml
parsing library is needed. The two best candidates are \texttt{ElementTree}
from the standard library or the extenal library \texttt{lxml}. They both have
very similiar syntax, and \texttt{ElementTree} is basically a subset of
\texttt{lxml}. I don't stricktly need any of the features from \texttt{lxml}
though it's handling of name spaces is better than \texttt{ElementTree}, so to
reduce the number of dependencies I have opted for \texttt{ElementTree}

\textbf{scraping}. There are many good python libraries that can simplify the for web-scraping.
\begin{itemize}
    \item \textbf{\texttt{mechanize}:} Which has a very natural and powerful API
    \item \textbf{\texttt{urllib}:} Part of the Python standard library, can make GET requests.
    \item \textbf{\texttt{requests}:} One of the most popular Python libraries,
        has better API than urllib, but is not part of the standard library
    \item \textbf{\texttt{Beautiful Soup}:} Prominent html Parsing library
\end{itemize}
As I have a commitment to keeping the number of dependencies low I will parse
urls and make requests using \texttt{urllib} from the standard library. But as
many web pages can be malformed I need a mature library, that parses html much
the same way as the browser, and have thus settled on \texttt{Beautiful Soup}.

\section{Code and File Structure}
\subsection{EPUB Files}
The Base Idea is that for each book type, the base skeleton of will be available
under \texttt{book\_templates}, for now only epub is available. 

\begin{lstlisting}[language=bash]
\begin{verbatim}
$ ls -lh book_templates/epub/*
-rwxr-xr-x  1 jcr  staff   934B Nov 22 12:23 content.opf
-rwxr-xr-x@ 1 jcr  staff    20B Jun  4 23:19 mimetype
-rwxr-xr-x@ 1 jcr  staff    50B Jun  4 23:19 page_style.css
-rwxr-xr-x@ 1 jcr  staff   404B Jun  4 23:19 page_template.xhtml
-rwxr-xr-x@ 1 jcr  staff   670B Jun  4 23:19 titlepage.xhtml
-rwxr-xr-x  1 jcr  staff   578B Nov 22 17:40 toc.ncx

META-INF:
-rwxr-xr-x@ 1 jcr  staff   225B Nov 22 19:06 container.xml
\end{verbatim}
\end{lstlisting}

\subsection{Static Files}
Here I will describe all the files that are hard coded, and thus not changed by the code.

\texttt{mimetype} specifies the EPUP mimetype (\texttt{application/epub+zip}).
\texttt{META-INF/container.xml} points out where the root file is located, it
is hard coded to point to \texttt{content.opf}. \texttt{page\_style.css}
contains the css to style the book. While I could allow the parser to also
scrape style sheets, I think it is better to hard code it as some pages may
have the style sheet cascade in such a way that they are not applied correctly,
when only a small portion of the web page is scraped.

\subsubsection{XHTML Files}
%TODO:

\subsubsection{content.opf}
The \texttt{content.opf} is an xml file with two important tags
\begin{itemize}
    \item \texttt{<metadata>}, such as cover image, author name and book title.
    \item \texttt{<manifest>}, a reference to every file that needs to be
        indexed, it is hard coded to have references to \texttt{titlepage.xhtml}, \texttt{page\_style.css} and \texttt{toc.ncx}. Each chapter scraped should result in an xhtml file with an entry here.
\end{itemize}

\subsubsection{toc.ncx}
%TODO:


\texttt{page\_style.css} are small files that will not be touched by the code as they 





`webook/webook.py` contains the heart of the project. It contains the class `EBook`
which contains all the code necessary for creating an ebook. It has a fairly
simple API, In order to make it work, one must subclass it and 'teach' it to
parse your favorite websites. This is done by overwriting the abstract `parse` method in
the child classes, and by calling `update` for each chapter/section of the
book.

While it may sound very scary to subclass EBook, the parser for parsing
`www.fanfiction.net/` (`FanFictionEBook`) is less than 30 lines of code. To
read the code for inspiration, see `webook/modules/fanfiction.py`


\subsection{webserver}
\subsection{webscraping}
\subsection{modules}

%-------------------------------------------------------------------------------
% CONCLUSION
%-------------------------------------------------------------------------------
\chapter{Conclusion}


\begin{thebibliography}{9}
  % type bibliography here
\end{thebibliography}

\end{document}

%bibliography 
% epub http://idpf.org/epub
% epub201 http://idpf.org/epub/201
% epub301 http://idpf.org/epub/301
% epub2to3 http://www.idpf.org/epub/30/spec/epub30-changes.html
% dangling_else https://en.wikipedia.org/wiki/Dangling_else 
% robots http://www.robotstxt.org/robotstxt.html
% py_mechanize % https://pypi.python.org/pypi/mechanize/
% java_htmlunit http://htmlunit.sourceforge.net/
% worm https://parahumans.wordpress.com/
% flask http://flask.pocoo.org/ 
